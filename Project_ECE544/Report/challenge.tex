\section{Challenges}
\subsection{Radar Imaging Primer}
Radar generates images of objects by localizing the cluster of point reflectors that forms the object. 3D space imaging requires mapping point reflectors to voxels in a spherical coordinates with its distance, azimuth angle, and elevation angle. Distance is measured through to round-trip Time-of-Flight (ToF) of reflected radar signal, while azimuth and elevation angles can be either obtained by the beam steering angle of phased array antennas or be estimated with Direction-of-Arrival (DoA) estimation algorithms such as beam-forming or Multiple Signal Classification (MUSIC). However unlike the extremely narrow width of light beams, the cone-shape radar beam with sidelobes cause interference and leakage among voxels, which smears generated images. Additionally, with a few centimeter resolution of distance measurement, a continuum of a large number of point reflector sums up and makes it an underdetermined problem to localize them. Besides, radar reflection tends to be more specular than the mostly scattering reflection at optical frequencies. In other words, reflection off smooth objects might mainly towards an angle away from the radar receiver and disappears in the image. Hence, "edge detection" in radar images is very different from that of pictures. Firstly, it needs to predict high frequency information with only low frequency data. Secondly, it needs to learn to fill up missing parts of the object. 

\subsection{Related Work}
Although, previous work has demonstrates classification of radar images [], and RF-Pose3D ~\cite{rfpose} ~\cite{rfpose3D} Convolutional Neural Network (CNN) that can recreate the human body skeletons by tracking 14 key points. 

Problem with radar images is different and not well studied 

Presents an architecture that leverages deep learning to sense using RF signals. Our architecture consists of a component that generates training example, a , and a sensing component that infers properties. We show how to build these components using 

\subsection{Model Selection}

\subsection{Dataset Availability}
Once the network is setup, it needs training data -i.e., it needs many labeled exp
Dataset
	Variation between systems 
	Experiment
	Processing

\subsection{3D Complexity}
3.3D
3D CNN % rf 3D pose	
3D GAN size complexity


\subsection{Evaluation Metrics}


