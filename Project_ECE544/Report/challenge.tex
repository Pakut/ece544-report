\section{Challenges}
1. 
The low resolution of range and angle estimation in radar imagaing appears as . Besides
 
model
Problem with radar images is different and not well studied 
	The low resolution
	specularity 

related work classification
 RF-Pose3D ~\cite{rfpose} ~\cite{rfpose3D} demonstrates Convolutional Neural Network (CNN) that can recreate the human body skeletons by tracking 14 key points. Because CNNs leverage local dependencies in the data, it significantly reduces the total number of weights to be learned. Considering this favorable property of CNN, we are going to implement this model first in our project. In contrast, we will concentrate more on classifying features of vehicles to infer the shape, orientation, and even velocity. We are planning to start with 2D images, which contains the distance and angle of objects within a horizontal plane, then we will try to extend to 3D images.
	
New application 


Presents an architecture htat leverages deep learning to sense using RF signals. Our architecture consists of a component that generates training example, a , and a sensing component that infers properties. We show how to build these components using 


2. data
Once the network is setup, it needs training data -i.e., it needs many labeled exp
Dataset
	Variation between systems 
	Experiment
	Processing

3.3D
3D CNN % rf 3D pose	
3D GAN size complexity

4. 
Evaluation



