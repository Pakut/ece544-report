\section{Discussion and Conclusion} \label{conclusion}
\textcolor{red}{
Analyze the results, summarize the findings and point out possible future directions}

In this project we demonstrate an adapt of conditional GAN to radar images, aiming to improve the tradition low resolution, incomplete, and incomprehensible radar images in a way that they clearly depict the boundary and orientation of objects. GAN makes a great candidate for this specific application because it can successfully estimate high resolution information only low resolution data. However, as not been widely studied with neural networks, there is almost no public radar image dataset available, especially for the application of autonomous cars. Therefore, we also designed a data transformation and augmentation approach that leverages well-established vision and CAD model databases to synthesize realistic radar images. Experiments of 2D radar images have successfully generated enhanced radar images with precise boundaries and orientation with an 180 degree ambiguity, which can be easily fixed with the sign of Doppler shift. However, our current trained model is limited to the angle of view of the training dataset. Therefore, the next following step would be synthesizing 3D radar images, which shares the angle of view of our radar imaging systems, so that it can potentially enhance real radar images. Of course, due to the discrepancy between our training dataset of synthesized images and our test dataset of real images, the performance will degrade. We can either fundamentally compensate for this gap by improving the authenticity of synthesized radar images through 3D model fitting and specular reflection modeling, or we can try to include real radar images in the training set with ground truth obtained through synced vision detection outputs. Last but not least, in our current experiment, we only try to enhance images of cars, and we definitely want to gradually include more and more common participants of traffic. Eventually, we are looking forward to achieving comparable imaging quality as LiDAR while exceeding it in reliability and velocity measurement, so that Millimeter-wave systems can play a deterministic role in both perception and connectivity of the vision of autonomous and smart transportation.        
