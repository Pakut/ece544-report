\subsection{Dataset Generation}
As analyzed in section \ref{challenges}, when exploiting the application of deep learning in the new field of radar images, the problem of dataset availability is inevitable. Between the two options of building our own dataset: collecting real radar images through experiments and processing synthesized radar reflection, we have to make a trade-off. Although synthesizing dataset might be the only feasible way to create a big enough training dataset, the compatibility of trained model to real radar images significantly depends on the closeness between synthesized  and real radar reflection. Luckily, the transmission, propagation, and reflection of radar waveform are thoroughly studied and can be precisely modeled and simulated. Furthermore, with the help of advanced electromagnetic simulation softwares such as FEKO we can even model the attenuation and specularity of various objects and surfaces. Therefore, a well-designed radar reflection synthesizing algorithm should be indistinguishable from read radar signal, so it is hopeful that the performance of a cGAN model trained with synthesized radar images should not degrade too much when we use real radar image for testing instead. We are also planning to mixing a smaller number of real radar images with the synthesized dataset for training and find the improvement of cGAN performance with respect to the portion of real and synthesized data.  

The radar image synthesizing process can be further separated into: scene generation, reflector modeling, radar signal simulation, and image processing. The last two steps are straightforward, while the challenge lies in creating a realistic scene and reflector model. There are two options: recovering from video recording of street scenes, such as the Cityscapes dataset ~\cite{cityscapes}, or composing with 3D CAD models of typical objects like cars, pedestrians from datasets such as the Deformable 3D Cuboid Model dataset and the 3D ShapeNets dataset ~\cite{3Ddata, shapenets}.

2D
	Mask R-CNN
		input: radar image
		groundtruth: mask
	pros: large dataset with car truck human 
	cons: No 3D info, no specularity
	3D CAD
		input: contour
		groundtruth:
	pros: small dataset, single element 
	cons: 3D shape info, specularity 

