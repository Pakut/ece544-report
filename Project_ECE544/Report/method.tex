\section{ Method}
\textcolor{red}{
Describe the overall method on how you solve the proposed problem, and a bit of original derivation that has some relevance to what you’re trying to accomplish
}

Problem statement: low blured images no boundary can be seen, specularity causes missing parts, no availbale dataset, 4D CNN NN complex.

Overview 3D, in the method, we say that we start with 2D version, and based on that build 3D. 
We are trying to use cGAN to generate higher resolution images from low resolution radar images, which should have a sharp and accurate boundary of the object. Also, the missing parts due to specualarity of reflection need to be feel up.  S

cGAN:
	Why cGAN:
	1. GAN is tranfering images from 1 domain to another, some application 
	2. There are similar application using GAN: Vankat's work of using face recognition of thermal images to train camera images under low illumination sharp boundaries orientation of faces.
	3. L1 does give sharp boundary as wanted. We are not sure what would be good loss function.
	Transition: cGAN has input, groundtruth, G, D
	
	Input: Low resolution radar images, because there is no available public dataset of mmWave radar images, and collecting a big enough dataset by ourselves is not possible. We synthesized radar images. This heat-map is fed into the network. 
	Groundtruth: 
	1. Size of 3D 	which makes the training phase very slow even when using GPUs. 
	2. 2D:
	3. 3D


The input to our problem is pre-processed radar data. First, using the raw data from the antenna array, a coarse heat-map of objects are generated. After some further processing. 

Why not raw data: (Goes to detail of Dataset jayden)
Radar imaging processing algorithms is a well establish feild for 70 years. and there is fruitless to try and learn them using machine learning.
So instead of  The reason why we did not choose the raw data as input is twofold. 